

The Tigres project is a next generation scientific workflow API designed with ease of composition, scalability and light-weight execution in mind. This paper will focus on the development of an early stage user interface that aims to accelerate the prototyping of workflows by providing real-time visual feedback during workflow construction along with a code generation feature. Most programmers recognize that an IDE is necessary, but not sufficient for effective workflow design. The bridge between the graphical and code generating aspects of the interface will be an active notepad application inspired by Google's real-time search bar. Using a minimal command set, a user can enter and modify workflows in the active notepad with instant graphical feedback. The result will be an interface that feels like an IDE, but provides the option of graphical interaction and compositional streamlining. 
