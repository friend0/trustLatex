\section{Conclusion}
The purpose of this research was to investigate the possibility of an interface that provides visual feedback and workflow composition capability while remaining minimally intrusive. For the initial test cases, ATT's Dot language was employed to output graphs generated by the execution of Tigres Workflows. To extend this capability, various possibilities for making graphs interactive and editable were explored. D3 has proven to be a very capable library for the construction of dynamic and interactive graphs. However, the challenge of editing and interpreting these edits in D3 remains. Additionally, users may find the flashy nature of the library distracting. Additional research has uncovered a method of applying a jQuery plugin to SVG output of Dot files which may hold the greatest promise moving forward.

Through careful and honest analysis of the existing graphical workflow tools, the need for a more direct connection between graphic and programming domains was discovered. By employing built in Python libraries to build GUI elements which can be executed alongside existing workflows, the focus on the IDE remains paramount while the interface is available on demand. This paradigm leverages the flexibility and feature set available while coding in a modern IDE while preserving the visual capability of other workflow tools. 

\label{sec:conc} 






