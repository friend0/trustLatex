\section{Introduction}
Researchers in computationally intensive sciences are faced with analyzing rapidly expanding data sets on increasingly parallelized machines. These datasets have grown so large that it is no longer practical for partner institutions to download and analyze this information locally. Further, scaling these analyses to HPC systems is difficult even for experts. This paradigm poses great challenges for scientists developing the complex workflows necessary to operate on this data, particularly those working on data-intensive research. Scientists building workflows for data intensive analysis currently have one of two options. They can choose to write their own scripts in a programming language of their choice, or they can employ a specialized workflow tool like Taverna \cite{taverna}, Pegasus\cite{pegasus}, or Kepler \cite{kepler}. A popular approach to dealing with data-intensive tasks has been to follow the MapReduce pattern for parallelization, merging, and splitting tasks - sometimes referred to as 'Scatter-Gather.' At present, writing scientific workflows using a MapReduce model with a Hadoop implementation can require a significant amount of programming expertise and custom code. Some difficulty arises in scientific workflow implementations stemming from the difference in the nature of typical MapReduce tasks in comparison to scientific functions. Additionally, writing scripts from scratch comes with a considerable amount of overhead due to the complexity of provenance tracking, parallel fault tolerance and monitoring features. Currently there is a significant separation between workflow implementations in programming languages, and those built using specialized GUI workflow tools. In light of these challenges, a tool that can simplify the implementation of common computing patterns while offering a high degree of interactivity is highly desirable.




\subsection{Background}

The goal of the Tigres project is to develop a next generation scientific workflow tool that addresses these concerns. Following the examples of the MapReduce model, in particular the Apache Hadoop implementation, Tigres is designed to be a highly portable and lightweight API to radically simplify the deployment of scientific workflows to diverse resource environments. Workflows in Tigres can be composed with the flexibility and ease of standard bearer technologies like C and Python while eliminating the overhead of writing custom workflow scripts. To use Tigres, a user simply needs to import the Tigres library into their project and compose their workflow using Tigres Templates. These templates represent the most commonly used computational patterns: Sequence, Parallel, Split and Merge. These templates are used to organize Tigres Tasks, the atomic unit of execution. Each Task object holds a function or executable, input values and input types. In constructing a Workflow using these Tigres types, a user has done everything that they need to invoke the advanced monitoring features of Tigres. Provenance tracking, fault-tolerance, execution-state and visualization tools become artifacts of Tigres' use, not overhead. This model allows analyses created on a desktop to be scaled and executed seamlessly on a diverse set of resources. 
\end




	While the importance of effective workflow tools is widely recognized, the process of API design has thus far been colored by designer bias and preconception. The 'User Centered Design' model has proven to be an effective tool for successful API development because of its persistent focus on user wants and needs. Currently, the Tigres project is focusing on a number of use cases from the biological sciences to cosmology. These diverse user groups offer a unique opportunity to retrieve useful feedback from highly domain specific sciences in structured review processes referred to as 'Scientist Centered Design' or SCD. User input is particularly valuable in the early stages of development to catch confusing features before they become deeply embedded in the system. Throughout the SCD review process an API gets distilled into a tool that mets user needs without sacrificing usability. 
	
	Despite high demand for capable workflow tools, existing applications have failed to gain widespread acceptance due to clunky interfaces, high costs of adoption and loose focus on the user. At its core, Tigres endeavors to be as minimally intrusive as possible. A researcher using Tigres in the IDE of their choice should find that this experience is indistinguishable from using their IDE on any other type of project. This is desirable on the one hand because users have expressed their discontent with confusing features and opaque interfaces. On the other hand, many users still appreciate the interaction that a well designed UI can provide. An effective interface has the potential to eliminate much of the menial coding involved in workflow composition and provide valuable visual feedback without feeling intrusive and cumbersome.
	
	 The focus of this paper is the design and implementation of an interface developed to engage users and accelerate workflow development. This work aims to be a proof of concept for a future integrated environment where users can move seamlessly between graphical and source code domains. Modifying a directed graph could provide an equivalent workflow representation in C or Python instantly, while modifications to the workflow in source code are reflected in a visual representation just as quickly.
