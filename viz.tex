\section{Visualizations}

In spite of the shortcomings of many graphical interfaces for workflow tools, visualizations are often the first step to understanding the semantics of a workflow. If an API is sufficiently difficult to learn or chronometrically expensive to adopt, users may tend to compose their workflows entirely in a graphical interface. Although the Tigres API aims to be exceptional in its ease of use and ability to quickly integrate into existing projects, graphical representation will still be an essential, if peripheral, tool for composition and execution monitoring.

To this end, a tool called Graphviz was employed for the initial production of workflow execution graphics. Graphviz is an open source tool built on top of several popular graphing engines including 'dot', 'neato', and 'sfdp'. 'Dot' is a tool built by ATT for representing complex networks. It provides edge drawing and placement algorithms that allow dot files to be specified quite simply. Additionally, 'dot' offers a wide variety of sub graphing and rank specifications for exhaustively specified graphs. The tool can output into a number of common image formats including SVG, making manipulation of 'dot' files possible in web pages. Currently, 'dot' graphs take the form of static output images. There is little {\em direct} support for graphical editing, though this is an area of interest for ATT engineers. 

Work on Tigres visualizations began with manual specifications of workflows in the 'dot' language. Individual code snippets for each Tigres template (Sequence, Parallel, Merge and Split) were implemented, and the underlying structure was analyzed in order to produce an algorithm that could take tigress workflows of arbitrary size and convert its constituent templates into a 'dot' graphic using Graphviz. Once this work was complete, it was possible to perform additional sub-graphing and clustering to render workflows in an intuitive and easy to read format. (put some pictures here?)

In order to make this visualization unobtrusive to the user, this functionality is built into the monitoring capabilities of Tigres. In other words, generation of workflow execution graphs is another artifact of using Tigres types to construct a workflow. Initially, this functionality has been limited to working on executions from memory. Future iterations of the Tigres graph module can work from log files which track Tigres execution on computing systems in real-time. Each task holds information pertaining to its current execution state, allowing for changing visuals depending on the current status of a task. 

While direct support for interactive graphics in 'dot' is not possible, web browsers offer the advantage of advanced visual capacity without additional software tools. The first choice for cross platform compatibility and visual flare was the D3\cite{d3} library in javascript. Other JavaScript libraries that have been considered for retooling to fit Tigres visualizations include JS Plumb\cite{jsPlumb}, GraphDracula\cite{graphDracula}, and Arbor JS' HalfViz\cite{halfViz} extension of the dot language, which is renderable in web pages. This system was initially attractive in its similar, but scaled down functionality to 'dot', but still suffers from the inability to edit. 

D3's focus is on data-driven documents; by binding visual elements in a web page to data elements it is possible to create dynamic graphs and animations. Additional motivation for implementing visualization and manipulation functionality in a browser was to conform to Tigres' goal being easily adoptable. It's expected that many users would prefer to stick to their editor or IDE of choice, so having this functionality in a web browser instead of a standalone application would be less intrusive to the user. Additionally, due to the dynamic nature of the tool it would be well suited to operate on changing data sets.

Several visualization schemes were used as test cases for D3. Most familiar is the 'weighted graph' which uses a charged particle and spring constant model to control the placement and motion of nodes in the graph. Altough this model is visually interesting and fun to interact with, it suffers from a 'hairball' even with large charge repulsion between nodes for graphs of non-trivial size. To address this concern, a 'hive' visualization paradigm was explored. Hives use axes to organize nodes and draw edges based on user defined metrics. A secondary motivation for this style of visualization was the difficulties posed when representing complex dependency networks in a workflow. The 'hairball' effect when representing graphs in 'dot' and D3 were compounded when edges showing their dependencies were added. In a hive format, these 'hairballs' took an orderly and visually pleasing form. (put a hive picture here?)
Although hive graphs solve the problem of data dependency visualization, they are not so easily understood in the context of execution graphs. 


 The D3 library offers a robust solution for web-based graphics. In many cases, however, it can tend toward the experimental or quixotic. Other avenues include the use of the xDot\cite{xDot} python library, which allows for rapid dot visualization without a web browser. Interactivity is still limited, but the possibility of onboard viewing and editing may prove worth the additional development. In subsequent research a method for making dot files rendered in SVG draggable and somewhat modifiable was discovered. At present this seems to be the most lucrative way forward, since it would require the least development to bring it to full functionality. This would include, but not be limited to including graph cluster scaling and node creation.This method relies upon the dot jQuery plugin \cite{dotJquery} for interactivity. 


\label{sec:viz}
